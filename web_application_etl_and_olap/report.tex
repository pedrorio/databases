\documentclass[titlepage]{article}
\usepackage[T1]{fontenc}
\usepackage[utf8]{inputenc}
\usepackage{indentfirst}
\usepackage{lipsum}
\usepackage{amsmath}
\usepackage[margin=1in]{geometry}
\usepackage[font=small,labelfont=bf]{caption}
\usepackage{hyperref}
\usepackage{fancyhdr}
\usepackage{tikz}
\usetikzlibrary{graphs}
\usetikzlibrary{chains}
\usetikzlibrary{positioning}
\usetikzlibrary{quotes}
\numberwithin{figure}{section}
\numberwithin{table}{section}
\usepackage{booktabs}
\usepackage{tikz-network}
\pagestyle{fancy}
\usepackage{physics}
\usepackage{mathrsfs}
\usepackage{enumerate}
\usepackage{enumitem}
\usepackage{subcaption}
\usepackage{newfloat}
\usepackage{multirow}
\usepackage{graphicx}
\usepackage{xcolor}

\setlength{\parindent}{0em}
\setlength{\parskip}{0em}

\definecolor{background-color}{rgb}{0.95,0.95,0.92}

\usepackage{minted}
\usemintedstyle{solarized-dark}


\newmintedfile[inputsql]{sql}{
linenos,
autogobble,
breaklines,
breakanywhere,
style=solarized-dark,
bgcolor=background-color
}

\newmintedfile[inputtext]{text}{
autogobble,
breaklines,
breakanywhere,
% style=solarized-dark,
bgcolor=background-color
}

\fancyhead{}
\lhead{\includegraphics[scale=0.13]{images/logo.png}}
\rhead{Information Systems and Databases \\ 2020/2021\footnotesize}

\setlength{\parindent}{0pt}
\setlength{\parskip}{1em}

\date{}

\title{
\includegraphics[scale=0.3]{images/logo.png}\\
\vskip 0.5em
\Huge Data Science and Engineering\\
\LARGE Information Systems and Databases\\
\vskip 2em
\large Database Project, Part 3\\
\vskip 2em
\large Group 37\\
\large 45 hours
\vskip 2em
\large a\_SIBD77L05\\
\large Professor Inês Filipe
}

\author{
Pedro Rio, 33\%\\
%, 45\%\\
\textit{MSc Data Science and Engineering}\\
\textit{Instituto Superior Técnico}\\
\textit{97241}
% 3 + 3 + 3 = 9
\and
Sebastião Maia Cerqueira, 33\%\\
%, 35\%\\
\textit{MSc Data Science and Engineering}\\
\textit{Instituto Superior Técnico}\\
\textit{97108}
% 3 + 3 + 1 = 7
\and
Sebastião Almeida, 33\%\\
%\%\\
\textit{MSc Data Science and Engineering}\\
\textit{Instituto Superior Técnico}\\
\textit{97115}
% 1 + 3 = 4
}

\begin{document}

    \maketitle
    \onecolumn

    \section*{Integrity Constraints}
    (IC1) For every transformer, pv must correspond to the voltage of the busbar identified by pbbid.
    \inputsql{integrity_constraints/ic1.sql}
    (IC2) For every transformer, sv must correspond to the voltage of the busbar identified by sbbid.
    \inputsql{integrity_constraints/ic2.sql}

    (IC5) For every analysis concerning a transformer, the name, address values cannot coincide
    with sname, saddress values of the substation where the transformer is located (i.e., gpslat
    and gpslong have the same values in transformer and substation).
    \inputsql{integrity_constraints/ic5.sql}

    \section*{View}
    \inputsql{view.sql}

    \section*{SQL}

    Who are the analysts that have analyzed every incident of element ‘B-789’?
    Since incidents can only be analysed once, then at most 1 supervisor can have analyzed
    every incident of a single element.

    \inputsql{queries/q1.sql}
    \inputtext{queries/a1.txt}

    Who are the supervisors that do not supervise substations south of Rio Maior (Portugal) Rio Maior coordinates: 39.336775, -8.936379 (cf. Google Maps)?

    \inputsql{queries/q2.sql}
    \inputtext{queries/a2.txt}

    What are the elements with the smallest amount of incidents?
    For this query, we include elements with no incidents, since those are obviously the ones with the smallest amount of incidents.

    \inputsql{queries/q3.sql}
    \inputtext{queries/a3.txt}

    How many substations does each supervisor supervise? (including supervisors that do not
    supervise any at the moment)

    \inputsql{queries/q4.sql}
    \inputtext{queries/a4.txt}

    \section*{Application Development}

    The web application was deployed to sigma and is available at
    \href{https://web.tecnico.ulisboa.pt/~pedrorio/sibd}{web.tecnico.ulisboa.pt/~pedrorio/sibd},
    a non-root route as that the root had already a web application running.
    The application was deployed through sftp to the \textit{web/sibs} folder,
    before executing the \textit{run.sh} script to add the necessary afs permissions
    and install the required python libraries.

    The web server is running apache http and there is no application server.
    The web server executes the appropriate cgi python script according to the user route,
    and the script makes the connection to the postgres database.
    As such, there is no direct contact between the client and the database.

    There are four namespaces in the application: substation, busbar, transformer and non\_line\_incident.
    Each one includes files to add and create non-existing models, as well as list, edit, update and delete
    models that already exist.
    Several jinja2 templates are used to generate the html to add new models and list and edit existing ones.
    Each file uses transactions where several models need to be manipulated and is protected against sql injection.

    \section*{Indices}

    To optimise a query to get the number of transformers with a given primary voltage
    by locality, two indexes need to be created.
    The first is a hash index on the primary voltage column of the substation table, as the primary voltage
    only needs to be constrained to a certain value (equality query), and the second is a b+tree index on the locality
    of the transformer table, which is meant to help with the group by clause.
    Finally, since the natural join in the query will use the primary key of the substation table, there will already be
    an index automatically created by most DBMS on the primary key (which happens to be composite), so there is already
    an index that can be used for index-joins, and no index needs to be created there.

    \inputsql{indices/transformer_locality_substation_pv.sql}

    For this query, we suggest the creation of a B+tree over the description column, using the INCLUDE clause to include
    the id attribute in the index.
    Our reasoning is the following:
    (instant, id) is already the primary key of the incident table, and, therefore, there should already be a (composite)
    index automatically created over those two attributes.
    It is vital that said index is a B+Tree index, since we will
    be doing a range query over that index to select the id of the records that fall inside the desired instant interval.
    If, for some reason, the index automatically created is NOT a B+Tree, then we also suggest the creation of a B+Tree
    index over instant, with the INCLUSION of the id attribute in the index record, to allow for an index-only scan.
    The selection over the description attribute, since it is a "like" clause that is specifically a prefix, will be
    essentially a range query.
    If we consider a B+tree, which will use alphabetical order, we just have to navigate to the first element,
    in alphabetical order that starts with the given prefix, and scan the index forward until the first element that starts
    with a prefix that is alphabetically after the prefix we want.
    That's why we suggest the creation of a B+tree over the
    description attribute, and we imagine the query being answered by doing an index-only scan on the (instant, id) index
    to retrieve the id of the records in the desired instant, simultaneously with an index-only scan on the description index
    (that, to recall, includes the id attribute) to retrieve the (description, id) records, followed and finished by
    an intersection on the id attribute of the result of the two index-only scans.

    A note on the INCLUDE clause: If our DBMS does not support INCLUDE (Postgres does), then we suggest composite indexes
    instead.
    It's better with the INCLUDE since we only want to query over one of the attributes, we just want the other attribute
    there to allow for an index-only scan, and the composite index has a heavier update computational burden than an index
    over just one attribute, with a small storage increase from including the second attribute (and that increase is also
    present in a composite index).

    \inputsql{indices/incident_id_description_instant.sql}

    \section*{Multidimensional Model}
    \inputsql{star_schema.sql}

\end{document}
